\begin{resumo}[Abstract]
 \begin{otherlanguage*}{english}
  The present work addresses the implementation of a flexible private cloud at the Lab Telecom of the University of Brasília (UnB), aiming to optimize the use of idle legacy servers to provide on-demand virtual machines. Given the limitation of only three licensed workstations for approximately 30 active students who require high-performance software, such as Ansys HFSS, Keysight ADS, AWR Microwave Office, and Altium Designer, this research proposes a solution based on OpenStack. The general objective is to provide isolated and personalized virtual machines for each student, ensuring scalability, optimized CPU and RAM usage, and simplified provisioning and removal for incoming or graduating students. To achieve this, the methodology includes reactivating and inventorying existing hardware , deploying the virtualization platform with Ubuntu Server and KVM/OpenStack , standardizing Windows base images with desired licenses and plugins , automating the VM lifecycle via scripts/APIs , managing resources elastically with CPU/RAM ballooning and fair-share policies , monitoring performance and usage , documenting and training users , and planning for future expansion. The solution seeks to transform discontinued servers into an elastic infrastructure capable of meeting the laboratory's demands and simplifying computational resource management.

   \vspace{\onelineskip}
 
   \noindent 
   \textbf{Key-words}: Cloud Computing. OpenStack. Virtualization. Resource Management.
 \end{otherlanguage*}
\end{resumo}
