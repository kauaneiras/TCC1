\chapter{Solução Proposta}

Este capítulo descreve detalhadamente a solução proposta para otimizar o uso de recursos computacionais no Laboratório Compartilhado de Ensino e Pesquisa em Telecomunicações (Lab Telecom) da Universidade de Brasília (UnB). A abordagem central envolve a implementação de uma nuvem privada baseada no OpenStack, utilizando servidores legados subutilizados e aplicando técnicas de provisionamento dinâmico de recursos para a entrega de um serviço IaaS (\textit{Infrastructure as a Service}). Serão apresentados o contexto do laboratório, os recursos de \textit{hardware} disponíveis, a arquitetura da nuvem proposta, os detalhes de implementação e configuração, bem como as estratégias para provisionamento dinâmico, gerenciamento de imagens de máquinas virtuais (VMs) e políticas de escalonamento. O objetivo é demonstrar como essa solução pode fornecer ambientes capazes de suprir a falta de múltiplos servidores dedicados para simulações complexas e programas de alto desempenho, como Ansys, Cadence e SolidWorks. A proposta visa fornecer múltiplos ambientes isolados e eficientes para os discentes e pesquisadores do laboratório, aproveitando a infraestrutura existente e garantindo a segurança e a otimização dos recursos.

\section{Contexto do Laboratório e Problemas}

O cenário atual do Lab Telecom apresenta desafios significativos para o desenvolvimento de atividades que demandam alto poder computacional. Com um corpo discente de aproximadamente 30 alunos ativos, e tendo contribuído com a formação de mais de 100 profissionais formados em Engenharia Eletrônica, Aeroespacial, Automotiva, de Energia e de Software, o laboratório dispõe de apenas 3 máquinas dedicadas para a execução de simulações complexas e programas de alto desempenho, como Ansys, Cadence e SolidWorks. Essa limitação de recursos computacionais resulta em longas filas de espera, restrição no acesso simultâneo e, consequentemente, um gargalo no progresso das pesquisas e projetos acadêmicos dos alunos.

Como parte de uma instituição pública, o laboratório enfrenta fluxo de caixa irregular, dependendo de editais de fomento que não garantem um fluxo financeiro constante. Essa realidade inviabiliza a contratação de nuvens públicas como alternativa para a alta demanda computacional. Adicionalmente, a natureza das pesquisas, que frequentemente envolvem o desenvolvimento de tecnologias patenteáveis, impõe uma barreira de segurança: o armazenamento de propriedade intelectual sensível em plataformas de terceiros. Esses obstáculos restringem a capacidade de expansão e a agilidade necessárias para a pesquisa de ponta.

Este cenário inspirou o desenvolvimento deste trabalho, que busca a otimização dos recursos computacionais existentes. O laboratório possui servidores legados que, embora ociosos e antigos, mantêm capacidade computacional relevante. A proposta deste trabalho é a criação de uma nuvem privada utilizando essa infraestrutura subutilizada. Tal iniciativa visa desenvolver um ambiente dinâmico, seguro e escalável, que fornecerá estações de trabalho virtuais e isoladas para cada aluno, garantindo a continuidade das pesquisas do Lab Telecom.

\section{Recursos Disponíveis}

A solução proposta será implementada utilizando a infraestrutura de hardware existente e subutilizada no Lab Telecom. Os recursos disponíveis consistem em 10 servidores Dell capazes de atender às necessidades de virtualização e computação em nuvem. Os modelos de servidores, levantados na Tabela \ref{tab:lab_equipamentos_parados} incluem: PowerEdge 1950 1U, PowerEdge R610 1U, PowerEdge 2950 2U, PowerEdge 2900 4U, PowerEdge R900 4U. Embora sejam modelos antigos, esses servidores podem ser reativados e possuem capacidade de processamento e memória RAM que, quando combinadas e virtualizadas, podem oferecer um ambiente robusto para as simulações exigidas.

\begin{table}[h]
    \centering
    \caption{Levantamento dos equipamentos em desuso no Laboratório}
    \label{tab:lab_equipamentos_parados}
    \begin{tabular}{|l|c|p{6cm}|}
        \hline
        \textbf{Equipamento} & \textbf{Qtde} & \textbf{Especificações e Descrição} \\ \hline \hline
        Dell PowerEdge 1950 1U & 1 & Processador Intel Xeon; suporte a DDR2/DDR3. \\ \hline
        Dell PowerEdge R610 1U & 2 & Processador Intel Xeon; suporte a DDR2/DDR3. \\ \hline
        Dell PowerEdge 2950 2U & 1 & Processador Intel Xeon; suporte a DDR2/DDR3. \\ \hline
        Dell PowerEdge 2900 4U & 3 & Processador Intel Xeon; suporte a DDR2/DDR3. \\ \hline
        Dell PowerEdge R900 4U & 3 & Processador Intel Xeon; suporte a DDR2/DDR3. \\ \hline
    \end{tabular}
\end{table}

Além dos servidores, o laboratório dispõe de 3 switches Cisco com cabos que permitem taxas de transferência de dados de até 1 Gigabit por segundo (1 Gbps), que serão utilizados para a interconexão da infraestrutura de rede da nuvem privada. Para garantir a integridade e a disponibilidade dos dados, será implementada uma configuração de RAID~6 em todos os servidores, proporcionando alta tolerância a falhas e segurança de dados, caso um disco venha a falhar. A pesar da importância do desempenho computacional para essa aplicação, a escolha do RAID~6 é justificada pela prioridade de segurança dos dados.

\section{Requisitos do Projeto}
Para assegurar que a arquitetura sugerida seja capaz de atender às demandas dos discentes e pesquisadores do Lab Telecom, foram estabelecidos requisitos técnicos:

\subsection{Requisitos Funcionais}
Os requisitos funcionais definem as funcionalidades essenciais que o sistema deve oferecer. Com base nas necessidades levantadas por meio de entrevistas com o coordenador e discentes do Lab Telecom, a plataforma de nuvem a ser implementada deverá ser capaz de:

\begin{enumerate}
\item Provisionar de Máquinas Virtuais (VMs) - Permitir que os alunos do laboratório criem instâncias de máquinas virtuais de forma autônoma e sob demanda, apenas inserindo seus dados em uma API. 
\item Painel de Controle (Self-Service) - Fornecer um painel de controle intuitivo onde administradores do sistema possam realizar o ciclo de vida completo de suas VMs: criar, iniciar, pausar, reiniciar e excluir.
\item Acesso a Recursos Externos - As VMs devem ser capazes de acessar a internet.
\item Suporte a Sistemas Operacionais: Garantir a instalação e o funcionamento de sistemas operacionais Microsoft Windows de 64 bits e distribuições do Linux.
\item Suportar Softwares de Simulação - A infraestrutura deve suportar a execução satisfatória de softwares de simulação utilizados pelo laboretório, sendo eles: Ansys HFSS, Keysight ADS, SolidWorks e Altium Designer 24.
\item Catálogo de Imagens de VM - Deve haver um catálogo de imagens pré-configuradas e otimizadas.
\item Armazenamento de Dados Persistente - O sistema deve permitir que os alunos criem e gerenciem volumes de armazenamento (discos virtuais) persistentes. Esses volumes devem poder ser anexados e desanexados das VMs, garantindo que os dados de projeto do aluno não sejam perdidos quando uma VM é encerrada ou recriada.
\item Perfis de Acesso Diferenciados - O sistema deve suportar diferentes níveis de permissão (roles), como "administrador", "professor" e "aluno", cada um com acesso a diferentes funcionalidades do painel de controle
\item Escalonamento de recursos automático - As máquinas virtuais devem receber CPU e RAM conforme a demanda, e devolver esse recurso quando não estiverem mais em uso.
\end{enumerate}

\subsection{Requisitos Não-Funcionais}
Os requisitos não-funcionais estabelecem os padrões de qualidade, as restrições operacionais e as características de desempenho do sistema.

\begin{enumerate}
\item Latência de Provisionamento: O tempo decorrido entre a solicitação de criação de uma nova VM por um aluno e o momento em que ela se torna totalmente acessível (via RDP/SSH) não deve exceder 5 minutos para imagens pré-configuradas.
\item Tolerância a Falhas: A falha de um único nó de computação (servidor que hospeda as VMs) não deve derrubar o serviço como um todo. O painel de controle deve permanecer acessível, e os alunos devem ser capazes de iniciar novas VMs nos nós restantes.
\item Recuperabilidade (Backup e Restore): O sistema deve ter uma rotina de backup automatizada para os componentes críticos da infraestrutura.
\item Isolamento de usuários: O sistema deve impor um isolamento de rede e recursos estrito entre os projetos de cada aluno, prevenindo que um usuário possa acessar, intencionalmente ou não, os dados e as VMs de outro.
\item Escalabilidade Horizontal: A arquitetura deve permitir a adição de novos servidores de computação à nuvem de forma simplificada, através da execução dos scripts Ansible, sem a necessidade de uma reconfiguração complexa ou tempo de inatividade prolongado.
\item Atualizações de Software: Deve ser possível aplicar atualizações de segurança e novas versões de pacotes nos servidores da infraestrutura de maneira simples e para todos os usuários através da imagem base.
\item Recuperabilidade: O sistema deve ter uma rotina diária de cópia de segurança (\textit{backup}) automatizada.
\end{enumerate}

\section{Arquitetura da Nuvem Privada Dinâmica}

A arquitetura da nuvem privada proposta é projetada para maximizar a utilização dos recursos de hardware existentes, ao mesmo tempo em que oferece um ambiente flexível e escalável para os usuários. A estrutura da nuvem será organizada em camadas, garantindo a separação lógica e a modularidade dos componentes.

Na base da arquitetura, encontram-se os servidores físicos Dell, que atuarão como nós de computação e armazenamento. Sobre esses servidores, será implementada a camada de virtualização, utilizando o KVM (\textit{Kernel-based Virtual Machine}) em conjunto com o Ubuntu Server. A escolha do KVM é justificada por ser um sistema de código aberto, evitando gastos com licenças proprietárias e dependência tecnológica de empresas privadas. Além disso, ao compará-lo com o Xen, o KVM mostrou desempenho superior, maior facilidade de uso e de gerenciamento. Sua integração profunda com o \textit{kernel} Linux torna-o eficiente para ambientes de virtualização de alto desempenho. Como sistema operacional hospedeiro, será utilizado o Ubuntu Server, por ser uma plataforma estável e amplamente suportada para a execução do KVM e dos serviços do OpenStack.

A camada de orquestração será provida pelo OpenStack, que atuará como o sistema operacional da nuvem, gerenciando e coordenando todos os recursos computacionais, de rede e de armazenamento. Os principais componentes do OpenStack que farão parte dessa arquitetura incluem:

\begin{itemize}
 \item \textbf{Keystone (Serviço de Identidade):} Responsável pela autenticação e autorização de usuários e serviços, garantindo a segurança do acesso à nuvem;
 \item \textbf{Nova (Serviço de Computação):} Gerenciará o ciclo de vida das máquinas virtuais, incluindo o provisionamento, agendamento e alocação de recursos;
 \item \textbf{Neutron (Serviço de Rede):} Proverá a conectividade de rede para as VMs, permitindo a criação de redes virtuais isoladas, roteamento e serviços de segurança;
 \item \textbf{Cinder (Serviço de Armazenamento em Bloco):} Oferecerá volumes de armazenamento persistentes para as VMs, garantindo a integridade e a disponibilidade dos dados;
 \item \textbf{Glance (Serviço de Imagem):} Gerenciará as imagens de máquinas virtuais, permitindo o armazenamento e a distribuição de imagens base para o rápido provisionamento de novas VMs;
 \item \textbf{Horizon (Painel de Controle):} Fornecerá uma interface gráfica intuitiva para que os usuários e administradores possam interagir com a nuvem, gerenciar seus recursos e monitorar o ambiente.
\end{itemize}

Os usuários se conectarão às suas máquinas virtuais por meio de protocolos de acesso remoto, como SSH (\textit{Secure Shell}) para acesso via linha de comando ou RDP (\textit{Remote Desktop Protocol}) para acesso gráfico, garantindo que cada aluno tenha seu ambiente de trabalho isolado e personalizado.

\section{Implementação e Configuração}

A implementação da nuvem privada dinâmica seguirá uma abordagem faseada, garantindo a correta instalação e configuração de cada componente. Os passos detalhados para a implantação incluem:

\begin{enumerate}
 \item \textbf{Preparação dos Servidores Físicos:} Instalação do Ubuntu Server e configuração de RAID~6 em cada servidor Dell (PowerEdge~2950, 2900, 1950, R900), garantindo redundância e integridade dos dados.
 \item \textbf{Instalação e Configuração do KVM:} Instalação do hipervisor KVM e otimização de parâmetros para cargas de trabalho intensivas.
 \item \textbf{Implantação do OpenStack:} Instalação distribuída dos serviços (Keystone, Glance, Nova, Neutron, Cinder, Horizon), com nó de controle dedicado e nós de computação/armazenamento distribuídos.
 \item \textbf{Configuração de Rede e Armazenamento:} Criação de redes virtuais no Neutron e integração do Cinder com o armazenamento físico configurado em RAID~6.
\end{enumerate}


\section{Provisionamento Dinâmico de Recursos}

O objetivo do provisionamento dinâmico é maximizar a eficiência dos servidores físicos, ajustando continuamente a alocação de memória RAM e capacidade de CPU, garantindo que aplicações recebam os recursos necessários conforme sua demanda. No caso do Lab Telecom, o principal uso do hardware é para simulações complexas e programas de alto desempenho, como Ansys, Cadence e SolidWorks. Ainda sim, a variação de uso do hardware é muito alta, já que podem haver apenas um aluno ativo ou vários. Por isso, é necessário implementar um sistema de provisionamento dinâmico que possa ajustar a alocação de recursos de forma automática, garantindo quando apenas um usuário estiver ativo, ele receba o máximo de recursos, e quando houver muitos usuários, os recursos sejam distribuídos de forma equitativa entre todos os usuários.

\begin{itemize}
\item \textbf{Ballooning} - Um driver de \textit{ballooning} será executado dentro de cada máquina virtual (VM). Quando um servidor físico começa a ter escassez de memória, ele infla o "balão" nas VMs que estão com memória ociosa, forçando-as a liberar essa memória para as VMs que mais precisam.

\item \textbf{Fair-Share Scheduling} - O hipervisor garantirá uma distribuição justa de ciclos de CPU entre todas as VMs. VMs com maior prioridade ou demanda receberão mais tempo de processamento, mas nenhuma VM será completamente privada de recursos.
\end{itemize}

\subsection{Gerenciamento de Imagens de VMs}

\begin{enumerate}
 \item \textbf{Discos Diferenciais:} Ao provisionar uma nova VM a partir de uma imagem base, um disco diferencial (\textit{overlay} ou \textit{copy-on-write}) será criado. Este disco armazenará apenas as alterações feitas pelo usuário na VM, enquanto a imagem base permanece inalterada.
 \item \textbf{Volumes Cinder Anexados:} Para dados críticos que exigem maior persistência, os usuários poderão anexar volumes de armazenamento persistentes providos pelo serviço Cinder, independentes do ciclo de vida da VM.
\end{enumerate}

O gerenciamento centralizado de imagens permitirá que os administradores criem e atualizem imagens base, garantindo que todos os usuários tenham acesso às versões mais recentes dos softwares e configurações.