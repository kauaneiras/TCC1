\begin{resumo}

O presente trabalho aborda a implementação de uma nuvem privada flexível no Lab Telecom da Universidade de Brasília (UnB), visando otimizar o uso de servidores legados ociosos para prover máquinas virtuais sob demanda. Diante da limitação de apenas três estações licenciadas para cerca de 30 alunos ativos que necessitam de softwares de alto desempenho, como Ansys HFSS, Keysight ADS, AWR Microwave Office e Altium Designer, a pesquisa propõe uma solução baseada em OpenStack. O objetivo geral é fornecer máquinas virtuais isoladas e personalizadas para cada estudante, garantindo escalabilidade, uso otimizado de CPU e RAM, e facilidade no provisionamento e remoção. Para tanto, a metodologia inclui a reativação e inventário do hardware existente, implantação da plataforma de virtualização com Ubuntu Server e KVM/OpenStack, padronização de imagens base Windows com licenças e plugins, automatização do ciclo de vida das VMs via scripts/APIs, gestão elástica de recursos com CPU/RAM \textit{ballooning} e políticas de \textit{fair-share}, monitoramento de desempenho e uso, documentação e treinamento, e planejamento para expansão futura. A solução busca converter os servidores descontinuados em uma infraestrutura elástica, capaz de atender às demandas do laboratório e simplificar a gestão de recursos computacionais.

 \vspace{\onelineskip}
    
 \noindent
 \textbf{Palavras-chaves}: Computação em Nuvem. OpenStack. Virtualização. Provisionamento Dinâmico.
\end{resumo}
