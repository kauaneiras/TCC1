\chapter{Fundamentação Teórica}
% Reinicia numeração de tabelas para este capítulo
\setcounter{table}{0}

Este capítulo reúne a fundamentação teórica necessária, cobrindo conceitos de computação em nuvem, virtualização, nuvens privadas, arquitetura do OpenStack e princípios de elasticidade e provisionamento dinâmico de recursos.

\section{Computação em Nuvem (\textit{Cloud Computing})}
\label{sec:cloud-computing}

\subsection{Definição e Paradigma}

A computação em nuvem (\textit{cloud computing}) é definida na publicação especial 800-145 pelo \textit{National Institute of Standards and Technology}(NIST), como um modelo que permite acesso a um conjunto compartilhado de recursos computacionais configuráveis, como redes, servidores, armazenamento, aplicações e serviços,  sob demanda e via rede. Esses recursos podem ser rapidamente provisionados e liberados com mínimo esforço de gerenciamento ou interação com o provedor de serviços \cite{mell2011}. Essa definição enfatiza características essenciais da computação em nuvem, dentre as quais destacam-se:

\begin{itemize}
    \item \textbf{Autoatendimento sob demanda (\textit{On-demand self-service})}: o próprio usuário pode provisionar recursos, como tempo de CPU ou armazenamento, sempre que necessário, sem intervenção humana junto ao provedor \cite{mell2011};
    \item \textbf{Acesso amplo à rede (\textit{Broad network access})}: os serviços são acessíveis via internet por meio de mecanismos padronizados, compatíveis com múltiplos dispositivos \cite{mell2011};
    \item \textbf{Agrupamento de recursos (\textit{Resource pooling})}: processamento, memória, armazenamento e largura de banda são agrupados e redistribuídos dinamicamente entre múltiplos inquilinos (\textit{multi-tenant}) conforme a demanda \cite{mell2011};
    \item \textbf{Elasticidade rápida (\textit{Rapid elasticity})}: a capacidade de hardware aparenta ser ilimitada, podendo aumentar ou diminuir de forma ágil — muitas vezes automática — para acompanhar variações de carga \cite{mell2011};
    \item \textbf{Serviço mensurado (\textit{Measured service})}: o uso de recursos é monitorado, controlado e relatado, garantindo transparência ao provedor e ao consumidor \cite{mell2011}.
\end{itemize}

\subsection{Modelos de Serviço}

Com base nessas características, o NIST classifica os serviços em três modelos:

\begin{description}
    \item[Software como Serviço (SaaS)] oferece aplicações completas hospedadas na nuvem e acessadas pela internet. O usuário não gerencia a infraestrutura subjacente \cite{mell2011};
    \item[Plataforma como Serviço (PaaS)] provê um ambiente na nuvem para que o consumidor implante aplicações próprias, sem administrar servidores, rede ou sistemas operacionais \cite{mell2011};
    \item[Infraestrutura como Serviço (IaaS)] disponibiliza recursos fundamentais — processamento, armazenamento e rede — permitindo ao usuário instalar sistemas operacionais e aplicações com controle limitado de rede \cite{mell2011}.
\end{description}



A computação em nuvem (\textit{cloud computing}) é definida pelo \textit{National Institute of Standards and Technology}(NIST), como um modelo que permite acesso a um conjunto compartilhado de recursos computacionais configuráveis, como redes, servidores, armazenamento, aplicações e serviços,  sob demanda e via rede. Esses recursos podem ser rapidamente provisionados e liberados com mínimo esforço de gerenciamento ou interação com o provedor de serviços \cite{mell2011}. Essa definição enfatiza características essenciais da computação em nuvem, dentre as quais destacam-se: 
\textbf{Autoatendimento sob demanda (\textit{On-demand self-service}):} o usuário pode provisionar recursos computacionais, como tempo de servidor e armazenamento em rede, conforme a necessidade, automaticamente e sem necessidade de interação humana com cada provedor de serviço.
\textbf{Acesso amplo à rede (\textit{Broad network access}):} Os serviços da nuvem podem ser acessados pela grande maioria de aparelhos que possuem acesso à internet, com interfaces padronizadas que facilitam a experiência dos usuários que já estão acostumados com algum serviço na nuvem.
 
\textbf{Agrupamento de recursos (\textit{Resource pooling}):} Os recursos computacionais, como processamento, memória, armazenamento e largura de banda de rede, são agrupados pelo provedor para atender múltiplos inquilinos usando um modelo de multilocação (\textit{multi-tenant}), com diferentes recursos sendo distribuídos dinamicamente e redistribuídos conforme a demanda do inquilino.
\textbf{Elasticidade rápida (\textit{Rapid elasticity}):} Os recursos podem ser provisionados e liberados elasticamente, em alguns casos automaticamente, para escalar rapidamente conforme a demanda. Para o usuário, as capacidades disponíveis para provisionamento parecem ser ilimitadas, a depender da quantidade de hardware que o provedor possui,  e podem ser utilizadas em qualquer quantidade, a qualquer momento. 
\textbf{Serviço mensurado (\textit{Measured service}):} Sistemas de nuvem controlam e otimizam automaticamente o uso de recursos utilizando recursos de medição. Devido a isso, o uso dos recursos pode ser monitorado, controlado e reportado, proporcionando transparência tanto para o provedor quanto para o consumidor do serviço utilizado. 

Além dessas características essenciais, os serviços oferecidos por provedores de nuvem são classificados em modelos de serviço:

\textbf{Plataforma como Serviço (\textit{PaaS})}: é um modelo que fornece ao consumidor a possibilidade de implantar, sobre a infraestrutura de nuvem, aplicações criadas ou adquiridas usando linguagens de programação, bibliotecas, serviços e ferramentas compatíveis com o ambiente oferecido pelo provedor. O consumidor não gerencia nem controla a infraestrutura subjacente da nuvem, como redes, servidores, sistemas operacionais ou armazenamento, mas possui controle sobre as aplicações implantadas e, possivelmente, sobre as configurações do ambiente onde essas aplicações serão hospedadas \cite{mell2011}.

\textbf{Infraestrutura como Serviço (\textit{IaaS})}: é um modelo que fornece ao consumidor a capacidade de provisionar recursos computacionais, como processamento, armazenamento, redes e outros, permitindo que ele implante e execute softwares, incluindo sistemas operacionais e aplicações. O consumidor também não gerencia nem controla a infraestrutura física da nuvem, mas tem controle sobre os sistemas operacionais, o armazenamento, as aplicações implantadas e, em alguns casos, controle limitado de componentes específicos de rede, como firewalls do host \cite{mell2011}.








 



Além dessas características essenciais, os serviços oferecidos por provedores de nuvem são classificados em modelos de serviço:

 





\begin{figure}[htb]
\centering
\includegraphics[width=0.8\textwidth]{figuras/Figura 1 - Comparação entre os modelos de serviço.png}
\caption{Comparação entre os modelos de serviço IaaS, PaaS e SaaS}
    \fonte{Adaptado de Rizvi et al. (2024).}
\label{fig:modelos-servico}
\end{figure}

Para facilitar o entendimento dos conceitos da Virtualização Total e Paravirtualização, a Tabela 1 compara os dois tipos. com suas vantagens e desvantagens: de serviço
 da nuvem. Em sistemas de SaaS, o software é hospedado na nuvem e acessível ao usuário pela internet. O usuário não possui nenhum controle sobre a aplicação, nem sobre a infraestrutura. Grande parte dos serviços oferecidos à usuários da internet utilizam esse modelo \cite{rizvi2024}. Dentre eles, pode-se listar, as redes sociais, como o \textit{Instagram}, \textit{TikTok} e \textit{Whatsapp}, e serviços de transmissão contínua de conteúdo multimídia (\textit{streaming}), como o \textit{Youtube}, \textit{Netflix} e \textit{Spotify}. O modelo PaaS, em contrapartida, fornece um ambiente de desenvolvimento e implantação pré-construído na nuvem, onde o usuário tem controle sobre a aplicação, incluindo códigos e bancos de dados. Nesse modelo encontram-se os serviços de desenvolvimento e hospedagem de software, como \textit{Microsoft Azure}, \textit{Google App Engine} e \textit{Heroku} \cite{rizvi2024}. Já no IaaS, o usuário pode solicitar uma quantidade de recursos, e tem controle sobre o sistema operacional. Nesse modelo estão as máquinas virtuais e serviços de armazenamento, como  \textit{Amazon Web Services (AWS) EC2 e S3},  \textit{Microsoft Azure Virtual Machines} e  \textit{Google Compute Engine}  \cite{rizvi2024}.

\subsection{Modelos de Implantação}

A implantação de sistemas de computação em nuvem pode ser realizada por meio de diferentes modelos. Cada modelo possui características peculiares que determinam sua adequação a contextos específicos. Os modelos mais conhecidos são a nuvem pública, privada, comunitária e híbrida, os quais se diferenciam pelo nível de controle, segurança, escalabilidade e custo de implantação \cite{mell2011}.

\begin{itemize}
\item \textbf{Nuvem Pública (\textit{Public cloud})}: As nuvem pública são gerenciadas por provedores terceirizados que disponibilizam recursos computacionais por meio da internet \cite{carroll2011}. Esses provedores costumam oferecer capacidades elevadas de escalabilidade e custos iniciais mais acessíveis, o que os torna atrativos para diferentes perfis de organizações \cite{amajuoyi2024}. Além disso, esse modelo de nuvem pode ser particularmente vantajoso para instituições que não contam com equipes internas especializadas no assunto ou que optam por não direcionar recursos à aquisição e manutenção de infraestrutura física própria. Em função desses benefícios, a adoção da nuvem pública tem apresentado um crescimento contínuo nos últimos anos \cite{amajuoyi2024}. Entretanto, as nuvens privadas podem apresentar desafios relevantes em termos de segurança e controle sobre a infraestrutura física. A nuvem pública é aberta a todos e sem garantia de alto nível de segurança, podendo trazer riscos à confidencialidade dos dados, à proteção de informações estratégicas e à perda ou roubo de dados, sendo uma opção mais adequada para empresas com "baixas preocupações de segurança" \cite{sathya2023}.

\item \textbf{Nuvem Privada (\textit{Private cloud})}: A infraestrutura é fornecida para uso exclusivo de uma única organização. Essa infraestrutura pode ser de propriedade, de terceiros, ou uma combinação de ambos, e pode estar localizada dentro ou fora das instalações da organização \cite{mell2011}. Para organizações que desejam um grau elevado de controle, personalização e alto nível de segurança, as nuvens privadas podem se destacar como uma solução adequada \cite{swapna2023}. Isso possibilita o desenvolvimento de um ambiente seguro e adaptado às exigências particulares de instituições que necessitam resguardar pesquisas sigilosas e proteger a propriedade intelectual. Dessa forma, a nuvem privada se mostra a alternativa ideal para operações que demandam um rigoroso gerenciamento de dados e confidencialidade, assegurando a integridade de informações estratégicas e o cumprimento de normativas internas.

\item \textbf{Nuvem Comunitária (\textit{Community cloud})}. A infraestrutura é fornecida para uso exclusivo de uma comunidade específica de clientes, pertencentes a diferentes organizações que compartilham interesses comuns. A infraestrutura pode ser de propriedade, de terceiros, ou uma combinação desses, e pode estar localizada dentro ou fora das instalações das organizações envolvidas \cite{mell2011}.

\item \textbf{Nuvem Híbrida (\textit{Hybrid cloud})}. Essa infraestrutura é uma composição de duas ou mais infraestruturas distintas (privada, comunitária ou pública)  \cite{mell2011}.

A implantação de sistemas de computação em nuvem pode ser realizada por meio de diferentes modelos, cada qual adequado a cenários específicos. Os modelos mais difundidos são as nuvens pública, privada, comunitária e híbrida, que se diferenciam em nível de controle, segurança, escalabilidade e custo de implantação \cite{mell2011}.

\begin{description}
  \item[\textbf{Nuvem Pública} (\textit{Public cloud})] \hfill \\ As nuvens públicas são gerenciadas por provedores terceirizados e disponibilizam recursos via internet \cite{carroll2011}. Costumam oferecer alta escalabilidade e baixos custos iniciais, favorecendo organizações de perfis variados \cite{amajuoyi2024}. Entretanto, por serem ambientes compartilhados, podem impor riscos à confidencialidade de dados e perda de controle sobre a infraestrutura, sendo mais recomendadas a empresas com menores exigências de segurança \cite{sathya2023}.

  \item[\textbf{Nuvem Privada} (\textit{Private cloud})] \hfill \\ É uma infraestrutura dedicada a uma única organização, podendo estar on-premises ou em instalações de terceiros \cite{mell2011}. Garante maior controle, personalização e níveis mais altos de segurança. Destaca-se em contextos que exigem proteção de propriedade intelectual ou dados sensíveis \cite{swapna2023}.

  \item[\textbf{Nuvem Comunitária} (\textit{Community cloud})] \hfill \\ Infraestrutura compartilhada por um conjunto de organizações com interesses comuns (por exemplo, requisitos regulatórios), podendo ser mantida interna ou externamente \cite{mell2011}.

  \item[\textbf{Nuvem Híbrida} (\textit{Hybrid cloud})] \hfill \\ Combina duas ou mais infraestruturas (pública, privada ou comunitária), permitindo portabilidade de dados e aplicações entre ambientes \cite{mell2011}.
\end{description}
 

\section{Virtualização e Hipervisores}
A virtualização é uma técnica de grande importância na computação moderna, uma vez  abstrai o hardware de máquinas físicas, agrupando logicamente os recursos de servidores. Esses agrupamentos lógicos, também conhecido como (\textit{pools}), permite a criação de máquinas virtuais (VMs) isoladas em um único sistema computacional físico (\cite{carissimi2008}; \cite{kominos2017}). Cada VM funciona como um ambiente de execução completo e autônomo, com sistema operacional próprio, aplicações, configurações e  serviços de rede, como se fosse uma máquina física independente (\cite{carissimi2008}; \cite{smith2005}). Esta capacidade de particionamento lógico de recursos de hardware é um dos fundamentos da computação em nuvem, viabilizando a elasticidade, a multilocação e a eficiência de custos (\cite{chawla2025}).

A virtualização é fundamental na computação moderna por abstrair o hardware de servidores físicos e agrupar logicamente seus recursos em \textit{pools}. Esse agrupamento permite a criação de máquinas virtuais (VMs) isoladas em um único host físico (\cite{carissimi2008}; \cite{kominos2017}). Cada VM opera como um sistema autônomo, com sistema operacional, aplicações e serviços de rede próprios, como se fosse um hardware independente (\cite{carissimi2008}; \cite{smith2005}). Esse particionamento lógico dos recursos viabiliza os pilares da computação em nuvem — elasticidade, multilocação e eficiência de custos (\cite{chawla2025}).

\subsection{Tipos de Virtualização}
A implementação de máquinas virtuais de sistema pode ser realizada através de duas técnicas principais: a virtualização total e a paravirtualização (\cite{carissimi2008}).
{{ ... }}

% 2.2.4 – KVM como Solução de Hipervisor
\subsection{KVM como Solução de Hipervisor}

O \textit{Kernel-based Virtual Machine} (KVM) é a solução de virtualização que integra um hipervisor diretamente ao \textit{kernel} do \textit{Linux} (\cite{carissimi2008}). Embora seja classificado como um hipervisor Tipo 2, pois opera sobre um sistema operacional, sua integração com o \textit{kernel} lhe confere características de desempenho e eficiência que se aproximam de soluções Tipo 1 (\cite{chawla2025}; \cite{kominos2017}). No KVM, cada máquina virtual (VM) é gerenciada como um processo regular do \textit{Linux}, agendado pelo escalonador padrão do sistema, o que otimiza a gestão de recursos e a alocação de memória (\cite{anand2013}).
Para minimizar a sobrecarga da tradução binária, o KVM explora as extensões de virtualização de hardware presentes nos processadores modernos, como Intel VT-x e AMD-V, (\cite{chawla2025}; \cite{carissimi2008}). Essa abordagem permite que o código do sistema operacional seja executado diretamente no processador, resultando em um desempenho muito próximo ao de uma máquina \textit{bare-metal} (\cite{kominos2017}). Estudos comparativos indicam uma sobrecarga do processador de apenas 4,0\% para o KVM, valor competitivo frente a outras soluções (\cite{chawla2025}). Além disso, em virtude de se integrar com o \textit{kernel} \textit{Linux}, o KVM se favorece das otimizações e avanços do próprio \textit{kernel}, como gerenciamento de memória, escalonamento de processos e suporte a hardware (\cite{anand2013}; \cite{arora2014}).
Apesar de suas vantagens, a segurança do KVM é intrinsecamente ligada ao kernel Linux, e dessa forma, vulnerabilidades no kernel podem expor o hipervisor a ataques como \textit{VM escape} ou \textit{hyperjacking} (\cite{chawla2025}). Soluções como o \textit{VMware ESXi} e o Xen possuírem um \textit{microkernel} próprio, e consequentemente, apresentam uma superfície de ataque menor. Em função disso, são frequentemente consideradas mais resilientes em ambientes corporativos que exigem conformidade com padrões de segurança rigorosos (\cite{chawla2025}).



Em contrapartida, a superfície de ataque do KVM está diretamente relacionada ao tamanho e à segurança do \textit{kernel} \textit{Linux}. Vulnerabilidades no \textit{kernel} podem levar a ataques como \textit{VM escape} ou \textit{hyperjacking}. Hipervisores com \textit{microkernel} próprio, como o \textit{VMware ESXi} e o Xen, apresentam uma base de código menor e, consequentemente, área de ataque reduzida, sendo preferidos em ambientes que demandam requisitos de segurança mais rigorosos (\cite{chawla2025}).
{{ ... }}
\section{Máquinas Virtuais \& Imagens}

% 2.3 – OpenStack
\section{OpenStack}

% 2.3.1 – Definição e Características
\subsection{Definição e Características}

O  \textit{OpenStack} é uma plataforma de software de código aberto (\textit{open-source}) utilizada para criar e gerenciar nuvens no modelo \textit{IaaS}. Sua história começou em 2010 como um projeto conjunto entre a  \textit{Rackspace Inc.} e a Administração Nacional da Aeronáutica e Espaço dos Estados Unidos \textit{NASA}. A NASA, buscando uma plataforma de nuvem robusta para suas vastas necessidades computacionais e de dados, desenvolveu a plataforma de computação "Nebula". Em paralelo, a Rackspace desenvolveu uma solução de armazenamento de objetos altamente escalável. As duas organizações combinaram suas tecnologias e lançaram o código-fonte como OpenStack sob uma licença  \textit{Apache 2.0}, permitindo que qualquer pessoa o utilizasse, modificasse e distribuísse livremente \cite{nasa2012}, estimulando assim a criação de uma comunidade global de desenvolvedores.

A fama do OpenStack deve-se principalmente à sua natureza aberta, flexibilidade e eficiência de custos. Por ser de código aberto, ele permite que empresas evitem altos custos de licenças proprietárias e dependência tecnológica. Sua arquitetura modular permite que os usuários implantem apenas os componentes desejados, como computação ( \textit{Nova}), armazenamento ( \textit{Swift}), rede ( \textit{Neutron}) e personalizem a infraestrutura para atender a requisitos específicos \cite{grzonka2015}. Essa adaptabilidade, combinada com a capacidade de escalar massivamente e o forte apoio de uma comunidade global e de grandes empresas, tornou o OpenStack a plataforma de escolha para uma vasta gama de organizações, desde data centers e provedores de telecomunicações até instituições de pesquisa como a Organização Europeia para a Investigação Nuclear (CERN) \cite{rousseau2019}.

% 2.3.2 – Módulos para configurações básicos do OpenStack
\subsection{Módulos para configurações básicos do OpenStack}

Por ser um conjunto de projetos de código aberto, o OpenStack oferece uma variedade de módulos que podem ser incluídos no projeto conforme a necessidade. Porém, alguns módulos básicos costumam aparecer na maior parte das soluções:

\begin{itemize}
    \item \textbf{Keystone: Serviço de Identidade}\\
    O \textit{Keystone} é o serviço de identidade do OpenStack, responsável por fornecer autenticação de clientes da API, descoberta de serviços e autorização distribuída multi-inquilino. Ele implementa a API de Identidade do OpenStack \cite{openstack2025}.\\
    Funcionalidades Principais:
    \begin{itemize}
        \item Autenticação: Gerencia a autenticação de usuários e serviços, permitindo o acesso aos recursos do OpenStack. Suporta diversos métodos de autenticação \cite{openstack2025}.
        \item Autorização: Controla o acesso aos serviços e recursos do OpenStack, definindo permissões para usuários e grupos \cite{openstack2025}.
        \item Catálogo de Serviços: Fornece um catálogo de todos os serviços do OpenStack disponíveis e seus respectivos endpoints (pontos de acesso), facilitando a descoberta de serviços pelos clientes \cite{openstack2025}.
        \item Gerenciamento de Usuários e Grupos (User and Group Management): Permite a criação e gerenciamento de usuários, grupos e projetos (tenants) dentro do ambiente OpenStack \cite{openstack2025}.
    \end{itemize}

    \item \textbf{Nova: Serviço de Computação}\\
    O \textit{Nova} é o módulo do OpenStack que provisiona instâncias de servidores virtuais. Ele suporta a criação de VMs e possui suporte para contêineres de sistema. O Nova opera como um conjunto de \textit{daemons} em servidores Linux existentes para fornecer esse serviço \cite{openstack2025}.\\
    Funcionalidades Principais:
    \begin{itemize}
        \item Gerenciamento de Instâncias: Permite a criação, inicialização, parada, reinicialização e exclusão de instâncias de computação.
        \item Agendamento de Recursos: O agendador do Nova determina em qual host de computação uma nova instância deve ser provisionada, com base em critérios como recursos disponíveis e políticas de afinidade.
        \item Gerenciamento de Ciclo de Vida: Gerencia o ciclo de vida completo das instâncias, desde o provisionamento até a terminação.
        \item Integração com Hipervisores: Suporta diversos hipervisores, incluindo KVM, Xen, VMware ESXi e Hyper-V, permitindo a execução de VMs em diferentes tecnologias de virtualização.
    \end{itemize}

    \item \textbf{Neutron: Serviço de Rede}\\
    O \textit{Neutron} é o módulo do OpenStack que fornece conectividade de rede entre dispositivos gerenciados pelo OpenStack, implementando uma API de Rede \cite{openstack2025}.\\
    Funcionalidades Principais:
    \begin{itemize}
        \item Gerenciamento de Redes (Network Management): Permite a criação e gerenciamento de redes virtuais, sub-redes e portas para as instâncias do OpenStack \cite{openstack2025}.
        \item Roteamento (Routing): Oferece capacidades de roteamento entre redes virtuais e para redes externas \cite{openstack2025}.
        \item Firewall como Serviço (Firewall as a Service - FWaaS): Permite a configuração de regras de firewall para proteger as instâncias e redes virtuais \cite{openstack2025}.
        \item Balanceamento de Carga como Serviço (Load Balancing as a Service - LBaaS): Distribui o tráfego de rede entre múltiplas instâncias para melhorar a disponibilidade e o desempenho \cite{openstack2025}.
        \item Rede Definida por Software (Software-Defined Networking - SDN): Integra-se com diversas tecnologias SDN para fornecer uma infraestrutura de rede flexível e programável \cite{openstack2025}.
    \end{itemize}

    \item \textbf{Horizon: Dashboard}\\
    O \textit{Horizon} é a interface gráfica (\textit{dashboard}) do OpenStack, que fornece uma visão unificada e uma ferramenta de gerenciamento para todos os serviços da plataforma \cite{openstack2025}. Através do Horizon, administradores podem monitorar recursos, criar e gerenciar instâncias, configurar redes, gerenciar volumes de armazenamento e realizar diversas tarefas operacionais de forma intuitiva, facilitando a administração da nuvem.\\
    Funcionalidades Principais:
    \begin{itemize}
        \item Interface Gráfica: Proporciona uma interface intuitiva para que usuários e administradores possam interagir com os serviços do OpenStack sem a necessidade de usar a linha de comando \cite{openstack2025}.
        \item Gerenciamento de Recursos: Permite gerenciar instâncias, redes, volumes de armazenamento, imagens e usuários de forma visual \cite{openstack2025}.
        \item Monitoramento: Oferece visibilidade sobre o status e o desempenho dos recursos da nuvem \cite{openstack2025}.
        \item Extensibilidade: É uma estrutura extensível, permitindo que desenvolvedores criem novos painéis e funcionalidades para atender a necessidades específicas \cite{openstack2025}.
    \end{itemize}

    \item \textbf{Glance: Serviço de Imagem}\\
    O \textit{Glance} atua como um serviço de registro e entrega de imagens de VMs. Ele permite que usuários armazenem e descubram imagens de disco de VM, que podem ser usadas como modelos para lançar novas instâncias \cite{openstack2024}.\\
    Funcionalidades Principais:
    \begin{itemize}
        \item Registro e Entrega de Imagens: Permite o upload, registro e descoberta de imagens de VMs \cite{openstack2024}.
        \item Armazenamento de Imagens: Armazena as imagens de VMs como sistemas de arquivos \cite{openstack2024}.
        \item Metadados de Imagens: Gerencia metadados associados às imagens, facilitando a busca e categorização \cite{openstack2024}.
        \item API RESTful: Oferece uma API RESTful para consulta de metadados de imagens e recuperação das imagens reais \cite{openstack2024}.
    \end{itemize}

    \item \textbf{Cinder: Serviço de Armazenamento em Bloco}\\
    O  \textit{Cinder} fornece armazenamento em bloco persistente para as instâncias OpenStack. Este módulo é crucial para garantir que todas as alterações feitas pelos usuários em suas máquinas virtuais sejam salvas. Cada máquina virtual de usuário teria um volume de boot do Cinder, garantindo que o sistema operacional e os dados do usuário permaneçam intactos mesmo após a VM ser desligada e reiniciada, atendendo ao requisito de ambiente persistente \cite{openstack2021}.\\
    Funcionalidades Principais:
    \begin{itemize}
        \item Criação e Gerenciamento de Volumes: Permite criar, anexar, desanexar e excluir volumes de armazenamento para as instâncias \cite{openstack2021}.
        \item Snapshots e Backups: Oferece a capacidade de criar  \textit{snapshots} de volumes para recuperação de dados e \textit{backups} \cite{openstack2021}.
        \item Tipos de Volume: Suporta diferentes tipos de volumes com características de desempenho e redundância variadas, permitindo que os usuários escolham o tipo de armazenamento mais adequado às suas necessidades \cite{openstack2021}.
        \item Integração com  soluções de terceiros: Pode se integrar com uma ampla gama de soluções de armazenamento de terceiros \cite{openstack2021}.
    \end{itemize}

    \item \textbf{Swift: Serviço de Armazenamento de Objetos}\\
    O  \textit{Swift} é um módulo para armazenamento de objetos altamente disponíveis. Organizações podem usar o Swift para armazenar grandes quantidades de dados de forma eficiente, segura e econômica \cite{openstack2023}.\\
    Funcionalidades Principais:
    \begin{itemize}
        \item Armazenamento de Objetos: Permite armazenar e recuperar objetos de dados de forma escalável e durável, sem a estrutura hierárquica de um sistema de arquivos tradicional \cite{openstack2023}.
        \item Alta Disponibilidade e Durabilidade: Garante a disponibilidade e a integridade dos dados através da replicação e distribuição de objetos em múltiplos nós de armazenamento  \cite{openstack2023}.
        \item Escalabilidade: Projetado para escalar horizontalmente, permitindo adicionar mais capacidade de armazenamento conforme a necessidade  \cite{openstack2023}.
        \item API RESTful: Oferece uma API RESTful simples para interagir com o serviço de armazenamento de objetos  \cite{openstack2023}.
    \end{itemize}
\end{itemize}







\section{Elasticidade e Provisionamento Dinâmico}

\subsection{Dimensionamento \textit{automático}}

\subsection{Políticas de \textit{Fair-Share} e \textit{Ballooning}}

\section{Imagens Base Compartilhadas}

\section{Trabalhos Relacionados}

\section{Síntese dos Conceitos}
