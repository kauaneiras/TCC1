\chapter*[Introdução]{Introdução}
\addcontentsline{toc}{chapter}{Introdução}


\section{Contextualização}

Nos últimos anos, a computação em nuvem se assentou como um modelo para o acesso flexível de recursos computacionais. De acordo com o National Institute of Standards and Technology (NIST), a computação em nuvem pode ser definida como um paradigma que possibilita o acesso sob demanda a um conjunto compartilhado de recursos de computação configuráveis. A ideia de concentrar o hardware e software "na nuvem", em outros termos, em data centers acessíveis remotamente, traz benefícios como o gerenciamento mais eficiente dos recursos computacionais \cite{mell2011}. Ao aproveitar mecanismos de multi-tenancy e elasticidade, as nuvens permitem que um cluster computacional seja compartilhado por diversos usuários, cada um isolado em um ambiente virtual próprio, enquanto a alocação de CPU, memória e armazenamento cresce ou encolhe automaticamente de acordo com a carga de trabalho. Essas técnicas adaptativas de alocação de recursos ajudam a manter a eficiência de utilização dos recursos em ambientes de nuvem \cite{dai2015, ray2013}. 

Organizações que buscam dinamismo, mas não desejam transferir dados sensíveis para provedores públicos, encontram em nuvens privadas uma solução viável. O uso de nuvens privadas permite que empresas mantenham controle sobre a segurança e a privacidade de suas informações, combinando a escalabilidade e a flexibilidade do ambiente em nuvem. Nesse contexto, destaca-se o OpenStack, um framework de código aberto que possibilita a construção de nuvens privadas escaláveis, com capacidade de atender demandas específicas de instituições acadêmicas e departamentos de engenharia que executam simulações de alto desempenho \cite{heuchert2021}. 

A computação em nuvem se caracteriza pelo modelo de multi-tenancy, onde múltiplos inquilinos compartilham a mesma infraestrutura física, ao mesmo tempo que detêm de um ambiente isolado de dados e aplicações. Essa abordagem pode ser um fator significativo para organizações que buscam maximizar a utilização de recursos disponíveis. O modelo de multi-tenancy permite que instituições acadêmicas e empresas atendam a distintas demandas e usuários simultaneamente, proporcionando uma maneira eficiente de gerenciar recursos computacionais mantendo isolamento de cada usuário e sistema \cite{heuchert2021}.

Em ambientes que exigem alto poder de processamento, como laboratórios de pesquisa onde discentes executam simulações científicas intensivas, a capacidade de compartilhar recursos computacionais de forma eficiente pode ter grande relevância. Xu, Zhang e Chen destacam a eficácia das plataformas de nuvem em simulações científicas intensivas, permitindo alocação otimizada de recursos computacionais e reduzindo custos operacionais \cite{xu2021}. Análises comparativas entre a computação em nuvem e os clusters convencionais indicam que instâncias em nuvem podem igualar ou até superar o desempenho de servidores potentes. A capacidade de alocar dinamicamente o processamento conforme a demanda torna essa alternativa interessante para ambientes acadêmicos e organizacionais com necessidade de alta performance e elasticidade computacional \cite{roloff2012}. 

\section{Justificativa}

\section{Objetivo Geral}

\section{Aspectos metodológicos}

\section{Estrutura da Dissertação}

