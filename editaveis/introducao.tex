% Introdução
\chapter{Introdução}

\section{Contextualização}

Nos últimos anos, a computação em nuvem se assentou como um modelo padrão para o acesso flexível de recursos computacionais através da rede. De acordo com o \textit{National Institute of Standards and Technology} (NIST), a computação em nuvem pode ser definida como um paradigma que possibilita o acesso sob demanda a um conjunto compartilhado de recursos de computação configuráveis. A ideia de concentrar o \textit{hardware} e \textit{software} na "nuvem", em outros termos, em \textit{data centers} acessíveis remotamente, traz benefícios como o gerenciamento mais eficiente dos recursos computacionais \cite{mell2011}. Atualmente, com um planeta cada vez mais digitalizado, a nuvem tem se tornado indispensável na vida de milhões de pessoas no mundo, tanto para a execução de tarefas cotidianas, como pedir uma comida em um aplicativo de \textit{delivery} e assistir um filme em plataforma de \textit{streaming}, quanto para fazer tarefas complexas, como pesquisas científicas que exigem alto uso de recursos computacionais.

As nuvens permitem que recursos computacionais sejam compartilhados por diversos usuários, cada um isolado em um ambiente virtual próprio, enquanto a alocação de unidade central de processamento (CPU), memória (RAM) e armazenamento cresce ou encolhe automaticamente de acordo com a carga de trabalho, tornando o uso dos recursos mais eficiente \cite{dai2015, ray2013}. Organizações que buscam dinamismo de recursos computacionais, alta disponibilidade, grandes quantidades de armazenamento e poder computacional, encontram nas nuvens uma solução viável, visto que existe um mercado competitivo de nuvens públicas com bons preços, dominado por grandes empresas, como o Google, Amazon e Microsoft. No entanto, para instituições que desejam manter controle rigoroso sobre o \textit{hardware}, a segurança e a privacidade de seus dados, nuvens privadas podem ser uma solução viável.

O modelo de nuvens privadas permite que as empresas usufruam dos benefícios de escalabilidade e flexibilidade inerentes aos ambientes de nuvem, preservando a soberania sobre suas informações e controle sobre o \textit{hardware}. Nesse cenário, o OpenStack emerge como um framework de código aberto que possibilita a construção de nuvens privadas robustas e escaláveis, capazes de atender as mais variadas demandas de instituições, incluindo realizar simulações em \textit{softwares} que exigem alto desempenho de \textit{hardware}, sendo assim, uma opção interessante a laboratórios de pesquisa e departamentos de engenharia \cite{heuchert2021}. A capacidade de distribuir recursos de uma única máquina a múltiplos usuários (\textit{multi-tenancy}), faz com que a nuvem seja uma alternativa interessante para instituições com número de usuários maior que o número de máquinas disponíveis.

No modelo de \textit{multi-tenancy}, usuários compartilham a mesma infraestrutura física, mas operam em ambientes virtuais isolados, sem compartilhamento direto de dados. Essa abordagem, somada à elasticidade de recursos, que se refere à capacidade de aumentar ou reduzir dinamicamente a alocação de CPU, memória e armazenamento conforme a carga de trabalho, maximiza a utilização dos recursos disponíveis, permitindo que instituições atendam a diversas demandas simultaneamente com menor quantidade de \textit{hardware}, ao mesmo tempo que mantêm o isolamento e a segurança de cada usuário e sistema \cite{heuchert2021}.

Para um ambiente acadêmico, onde o número de usuários e a intensidade das cargas de trabalho podem variar significativamente ao longo do dia ou do semestre, a possibilidade de escalonar automaticamente a CPU, memória e armazenamento é de extrema importância. Ao garantir que a nuvem se adapte dinamicamente às flutuações de demanda dos usuários, é possível fornecer mais recursos a um usuário com demandas mais altas e menos recursos para um usuário com demandas mais baixas \cite{armbrust2010; oliveira2015}, otimizando a distribuição de carga de trabalho do sistema e melhorando a experiência do usuário. Ademais, a capacidade de alocação dinamica de recursos pode tornar o desempenho de instâncias em nuvem superior a clusters convencionais \cite{roloff2012}, destacando a nuvem como uma excelente alternativa para instituições que desejam otimizar o uso de recursos computacionais.

\section{Justificativa}

O Laboratório de Telecomunicações (Lab Telecom) da Universidade de Brasília (UnB) é um centro de excelência que tem contribuído significativamente para a formação de mais de 100 alunos. Atualmente, o laboratório atende cerca de 30 alunos ativos que frequentemente precisam executar softwares de alto desempenho, como Ansys HFSS, Keysight ADS, AWR Microwave Office e Altium Designer. No entanto, a infraestrutura atual do laboratório, que dispõe de apenas três estações licenciadas, gera filas e atrasos nas atividades acadêmicas, comprometendo a produtividade e o avanço das pesquisas.

Diante desse cenário, a reativação de servidores descontinuados e sua conversão em uma nuvem privada, baseada em imagens padronizadas desses aplicativos, surge como uma solução inovadora e altamente eficaz. Essa abordagem permitirá que cada estudante tenha acesso a uma máquina virtual (VM) própria, isolada e personalizada, disponível sob demanda. Este modelo não apenas otimiza o uso de hardware ocioso, mas também confere elasticidade à infraestrutura: quando um único usuário está ativo, ele pode usufruir de 100\% dos recursos de CPU e RAM, e, à medida que mais alunos acessam o sistema, os recursos são automaticamente repartidos. Além disso, a gestão do ciclo de vida das VMs é simplificada, permitindo o descarte rápido de máquinas de alunos que concluíram seus estudos e a criação ágil de novas VMs para discentes ingressantes. Essa solução visa, portanto, não só resolver o problema imediato de acesso a recursos, mas também simplificar a expansão futura do laboratório com a adição de novos servidores.

\section{Objetivo Geral}

Implementar uma nuvem privada flexível que utilize os servidores ociosos do Lab Telecom para entregar, sob demanda, máquinas virtuais capazes de executar softwares licenciados para o maior número de alunos possível, garantindo escalabilidade, uso otimizado de CPU e RAM, além de provisionamento e remoção facilitados.

\subsection{Objetivos Específicos}

\begin{enumerate}
    \item Reativar o hardware: Inventariar e restaurar os servidores legados, criando arranjos RAID nos hosts para aumentar a resiliência;
    \item Implantar a plataforma de virtualização: Instalar Ubuntu Server com KVM e OpenStack, habilitando serviços Nova, Neutron, Glance e configurando o Keystone. Desenvolver playbooks Ansible que adicionem novos nós ao cluster em poucos cliques;
    \item Padronizar imagens: Construir e manter uma imagem base do Windows com todas as licenças e plugins dos softwares desejados;
    \item Automatizar o ciclo de vida das VMs: Criar scripts/APIs que provisionem VMs para cada aluno ingressante, e scripts/APIs que arquivem ou excluam VMs de usuários que encerraram seu vínculo com o laboratório;
    \item Gerir recursos de forma elástica: Ativar \textit{ballooning} para CPU e RAM, e políticas de \textit{fair-share}, permitindo o uso exclusivo de 100\% dos recursos quando houver apenas um usuário, e a divisão dinâmica conforme a quantidade de VMs ativas;
    \item Monitorar desempenho e uso: Implementar dashboards com métricas de CPU, RAM, E/S, rede e licenças, além de alertas preventivos para saturação ou falhas de hardware;
    \item Documentar: Produzir manuais e guias para os alunos, reduzindo a curva de aprendizado e as solicitações de suporte;
    \item Planejar a expansão futura: Definir um checklist para integrar novos servidores (rede, storage, energia) e uma política de atualização periódica das imagens.
\end{enumerate}

\section{Organização do Trabalho}

O presente trabalho está estruturado em oito capítulos, organizados da seguinte forma:

\begin{itemize}
    \item \textbf{Capítulo 2 — Fundamentação Teórica}: apresenta os conceitos fundamentais de computação em nuvem, virtualização, hipervisores, Máquinas Virtuais, OpenStack, Ubuntu Server, RAID,  elasticidade e provisionamento dinâmico de recursos, Imagens Compartilhadas e Ansible;
    \item \textbf{Capítulo 3 — Solução Proposta}: descreve o problema enfrentado pelo Lab Telecom e a proposta da nuvem privada dinâmica, com os componentes envolvidos, os fluxos de provisionamento automático de VMs e as políticas de escalonamento adotadas;
    \item \textbf{Capítulo 4 — Metodologia}: explica a implementação prática da infraestrutura no Lab Telecom, incluindo a preparação dos servidores físicos, a instalação e configuração dos serviços do OpenStack e a criação das redes e imagens;
    \item \textbf{Capítulo 5 — Desenvolvimento do Mecanismo de Provisionamento Dinâmico}: detalha os scripts, APIs e automações que criam, ajustam e encerram máquinas virtuais conforme a demanda dos discentes;
    \item \textbf{Capítulo 6 — Implementação}: descreve o passo a passo seguido para implementar a nuvem privada dinâmica no Lab Telecom;
    \item \textbf{Capítulo 7 — Análise de Resultados}: discute os resultados obtidos, analisando o desempenho da solução, identificando vantagens, limitações e correlacionando-os aos objetivos inicialmente traçados;
    \item \textbf{Capítulo 8 — Conclusão}: encerra o trabalho com as conclusões gerais, revisita o atendimento dos objetivos, aponta as principais contribuições e sugere possíveis melhorias e extensões para pesquisas futuras.
\end{itemize}
