\chapter{Introdução}


\section{Contextualização}

Nos últimos anos, a computação em nuvem se assentou como um modelo para o acesso flexível de recursos computacionais. De acordo com o National Institute of Standards and Technology (NIST), a computação em nuvem pode ser definida como um paradigma que possibilita o acesso sob demanda a um conjunto compartilhado de recursos de computação configuráveis. A ideia de concentrar o hardware e software "na nuvem", em outros termos, em data centers acessíveis remotamente, traz benefícios como o gerenciamento mais eficiente dos recursos computacionais \cite{mell2011}. Ao aproveitar mecanismos de multi-tenancy e elasticidade, as nuvens permitem que um cluster computacional seja compartilhado por diversos usuários, cada um isolado em um ambiente virtual próprio, enquanto a alocação de CPU, memória e armazenamento cresce ou encolhe automaticamente de acordo com a carga de trabalho. Essas técnicas adaptativas de alocação de recursos ajudam a manter a eficiência de utilização dos recursos em ambientes de nuvem \cite{dai2015, ray2013}. 

Organizações que buscam dinamismo, mas não desejam transferir dados sensíveis para provedores públicos, encontram em nuvens privadas uma solução viável. O uso de nuvens privadas permite que empresas mantenham controle sobre a segurança e a privacidade de suas informações, combinando a escalabilidade e a flexibilidade do ambiente em nuvem. Nesse contexto, destaca-se o OpenStack, um framework de código aberto que possibilita a construção de nuvens privadas escaláveis, com capacidade de atender demandas específicas de instituições acadêmicas e departamentos de engenharia que executam simulações de alto desempenho \cite{heuchert2021}. 

A computação em nuvem se caracteriza pelo modelo de multi-tenancy, onde múltiplos inquilinos compartilham a mesma infraestrutura física, ao mesmo tempo que detêm de um ambiente isolado de trabalho e sem compartilhamento de dados. Essa abordagem pode ser um fator significativo para organizações que buscam maximizar a utilização de recursos disponíveis. O modelo de multi-tenancy permite que instituições acadêmicas e empresas atendam a distintas demandas e usuários simultaneamente, proporcionando uma maneira eficiente de gerenciar recursos computacionais mantendo isolamento de cada usuário e sistema \cite{heuchert2021}.

Outro ganho obtido da computação em nuvem é a elasticidade de recursos, que refere -se à capacidade de aumentar ou reduzir a alocação de recursos dinamicamente, aumentando ou diminuindo CPU, memória e armazenamento, de acordo com a quantidade de cargas de trabalho. A elasticidade faz com que a infraestrutura se adapte a picos de demanda, proporcionando aumento temporário na alocação de recursos \cite{armbrust2010}. Essa característica certifica que durante períodos de alta demanda, aplicações poderão receber recursos computacionais em níveis adequados à necessidade \cite{oliveira2015}, otimizando o throughput do sistema ao permitir que a capacidade ociosa seja redistribuída, e consequentemente, melhorando a experiência do usuário.

Em ambientes que exigem alto poder de processamento, como laboratórios de pesquisa onde discentes executam simulações científicas intensivas, a capacidade de compartilhar recursos computacionais de forma eficiente é pode ter grande relevância. Zhang e Zhang destacam a eficácia das plataformas de nuvem em simulações científicas intensivas, permitindo alocação otimizada de recursos computacionais e reduzindo custos operacionais \cite{xu2021}. Análises comparativas entre a computação em nuvem e os clusters convencionais indicam que instâncias em nuvem podem igualar ou até superar o desempenho de servidores potentes. A capacidade de alocar dinamicamente o processamento e a memória conforme a demanda torna essa alternativa interessante para ambientes acadêmicos e organizacionais com necessidade de alta performance e elasticidade computacional \cite{roloff2012}. 

O provisionamento dinâmico de máquinas virtuais (MVs) em nuvens privadas é uma estratégia prática de elasticidade e escalabilidade capaz de suprir a crescente demanda por eficiência, segurança e controle. O OpenStack, um dos frameworks de nuvem privada mais utilizados, suporta esse provisionamento dinâmico, permitindo que as organizações ajustem suas infraestruturas virtuais em resposta a variações nas cargas de trabalho \cite{callegati2016}. Por meio de algoritmos para alocação e distribuição de máquinas virtuais, é viável criar uma infraestrutura para laboratórios onde múltiplos usuários precisam ter máquinas virtuais individuais para simulações de alto desempenho, onde os recursos são utilizados de forma inteligente, alocando e desalocando recursos conforme a necessidade.

\section{Justificativa}
O Lab Telecom é um laboratório de excelência da Universidade de Brasília (UnB) que contribuiu com a formação de mais de 100 alunos desde a sua existência. Atualmente atende cerca de 30 alunos ativos que frequentemente precisam rodar softwares de alto desempenho, como o Ansys HFSS, Keysight ADS, AWR Microwave Office e Altium Designer. Porém, o laboratório dispõe de apenas três estações licenciadas, criando filas e atrasos nas atividades acadêmicas. Ao reativar servidores descontinuados e convertê-los em uma nuvem privada baseada em imagens padronizadas desses aplicativos, cada estudante poderá receber uma máquina própria isolada e personalizada, sob demanda.

Esse modelo torna a infraestrutura elástica, onde quando um único usuário está ativo, toda a CPU e RAM ficam à disposição dele, mas conforme mais alunos entram, os recursos são repartidos automaticamente. Quando um discente se forma, sua VM é simplesmente descartada, liberando espaço. Na chegada de um novo discente, cria-se uma nova máquina em poucos segundos. Além de aproveitar hardware ocioso, essa solução visa simplificar a expansão com mais servidores futuros.

\section{Objetivo Geral}
Implantar uma nuvem privada flexível que utilize os servidores ociosos do Lab Telecom para entregar, sob demanda, máquinas virtuais capaz de executar softwares licenciados que o laboratório possui para o maior número de alunos possível, garantindo escalabilidade, uso otimizado de CPU e RAM, provisionamento e remoção facilitadas para a entrada ou formatura dos discentes.

\subsection{Objetivos Específicos}
\begin{enumerate}
    \item \textbf{Reativar o hardware}
    \begin{itemize}
        \item Inventariar e restaurar os servidores legados.
        \item Criar arranjos RAID nos hosts para aumentar resiliência e desempenho de I/O.
    \end{itemize}
    
    \item \textbf{Implantar a plataforma de virtualização}
    \begin{itemize}
        \item Instalar Ubuntu Server com KVM/OpenStack, habilitando serviços-núcleo (Nova, Neutron, Glance) e configurando o Keystone.
        \item Desenvolver playbooks Ansible que adicionem novos nós ao cluster em poucos cliques.
    \end{itemize}
    
    \item \textbf{Padronizar imagens}
    \begin{itemize}
        \item Construir e manter uma imagem-base Windows com todas as licenças, plugins dos softwares desejados.
    \end{itemize}
    
    \item \textbf{Automatizar o ciclo de vida das VMs}
    \begin{itemize}
        \item Criar scripts/APIs que provisionam VMs para cada aluno ingressante.
        \item Criar scripts/API para arquivar ou excluir VMs de usuários que encerraram seu vínculo com o laboratório.
    \end{itemize}
    
    \item \textbf{Gerir recursos de forma elástica}
    \begin{itemize}
        \item Ativar CPU/RAM ballooning e políticas de fair-share, permitindo:
        \begin{itemize}
            \item uso exclusivo de 100\% dos recursos quando houver só um usuário;
            \item divisão dinâmica de CPU e memória conforme a quantidade de VMs ativas.
        \end{itemize}
    \end{itemize}
    
    \item \textbf{Monitorar desempenho e uso}
    \begin{itemize}
        \item Implementar dashboards com métricas de CPU, RAM, I/O, rede e licenças, além de alertas preventivos para saturação ou falhas de hardware.
    \end{itemize}
    
    \item \textbf{Documentar e treinar}
    \begin{itemize}
        \item Produzir manuais operacionais para a TI e guias rápidos para os alunos, reduzindo a curva de aprendizado e as solicitações de suporte.
    \end{itemize}
    
    \item \textbf{Planejar a expansão futura}
    \begin{itemize}
        \item Definir um checklist para integrar novos servidores (rede, storage, energia) e uma política de atualização periódica das imagens à medida que evoluem as versões dos softwares.
    \end{itemize}
\end{enumerate}

\section{Aspectos metodológicos}

\section{Organização do Trabalho}
O conteúdo do trabalho é dividido em 8 capítulos:
\begin{itemize}
    \item O capítulo 2 reúne a fundamentação teórica necessária, cobrindo conceitos de computação em nuvem, virtualização, nuvens privadas, arquitetura do OpenStack e princípios de elasticidade e provisionamento dinâmico de recursos.
    
    \item O capítulo 3 descreve detalhadamente a solução proposta, apresentando a arquitetura da nuvem privada dinâmica, os componentes envolvidos, os fluxos de provisionamento automático de VMs e as políticas de escalonamento adotadas.
    
    \item O capítulo 4 explica a implementação prática da infraestrutura no LabTelcom, incluindo a preparação dos servidores físicos, a instalação e configuração dos serviços do OpenStack e a criação das redes, imagens e flavours utilizados.
    
    \item O capítulo 5 aborda o desenvolvimento do mecanismo de provisionamento dinâmico, detalhando os scripts, APIs e automações que criam, ajustam e encerram máquinas virtuais conforme a demanda dos discentes.
    
    \item O capítulo 6 apresenta os cenários de teste definidos, a execução dos experimentos no ambiente real e a coleta de dados sobre tempo de instância, uso de recursos e persistência de dados dos usuários.
    
    \item O capítulo 7 discute os resultados obtidos, analisando o desempenho da solução, identificando vantagens, limitações e correlacionando-os aos objetivos inicialmente traçados.
    
    \item O capítulo 8 encerra o trabalho com as conclusões gerais, revisita o atendimento dos objetivos, aponta as principais contribuições e sugere possíveis melhorias e extensões para pesquisas futuras.
\end{itemize}
